
% Default to the notebook output style

    


% Inherit from the specified cell style.




    
\documentclass[11pt]{article}

    
    
    \usepackage[T1]{fontenc}
    % Nicer default font (+ math font) than Computer Modern for most use cases
    \usepackage{mathpazo}

    % Basic figure setup, for now with no caption control since it's done
    % automatically by Pandoc (which extracts ![](path) syntax from Markdown).
    \usepackage{graphicx}
    % We will generate all images so they have a width \maxwidth. This means
    % that they will get their normal width if they fit onto the page, but
    % are scaled down if they would overflow the margins.
    \makeatletter
    \def\maxwidth{\ifdim\Gin@nat@width>\linewidth\linewidth
    \else\Gin@nat@width\fi}
    \makeatother
    \let\Oldincludegraphics\includegraphics
    % Set max figure width to be 80% of text width, for now hardcoded.
    \renewcommand{\includegraphics}[1]{\Oldincludegraphics[width=.8\maxwidth]{#1}}
    % Ensure that by default, figures have no caption (until we provide a
    % proper Figure object with a Caption API and a way to capture that
    % in the conversion process - todo).
    \usepackage{caption}
    \DeclareCaptionLabelFormat{nolabel}{}
    \captionsetup{labelformat=nolabel}

    \usepackage{adjustbox} % Used to constrain images to a maximum size 
    \usepackage{xcolor} % Allow colors to be defined
    \usepackage{enumerate} % Needed for markdown enumerations to work
    \usepackage{geometry} % Used to adjust the document margins
    \usepackage{amsmath} % Equations
    \usepackage{amssymb} % Equations
    \usepackage{textcomp} % defines textquotesingle
    % Hack from http://tex.stackexchange.com/a/47451/13684:
    \AtBeginDocument{%
        \def\PYZsq{\textquotesingle}% Upright quotes in Pygmentized code
    }
    \usepackage{upquote} % Upright quotes for verbatim code
    \usepackage{eurosym} % defines \euro
    \usepackage[mathletters]{ucs} % Extended unicode (utf-8) support
    \usepackage[utf8x]{inputenc} % Allow utf-8 characters in the tex document
    \usepackage{fancyvrb} % verbatim replacement that allows latex
    \usepackage{grffile} % extends the file name processing of package graphics 
                         % to support a larger range 
    % The hyperref package gives us a pdf with properly built
    % internal navigation ('pdf bookmarks' for the table of contents,
    % internal cross-reference links, web links for URLs, etc.)
    \usepackage{hyperref}
    \usepackage{longtable} % longtable support required by pandoc >1.10
    \usepackage{booktabs}  % table support for pandoc > 1.12.2
    \usepackage[inline]{enumitem} % IRkernel/repr support (it uses the enumerate* environment)
    \usepackage[normalem]{ulem} % ulem is needed to support strikethroughs (\sout)
                                % normalem makes italics be italics, not underlines
    

    
    
    % Colors for the hyperref package
    \definecolor{urlcolor}{rgb}{0,.145,.698}
    \definecolor{linkcolor}{rgb}{.71,0.21,0.01}
    \definecolor{citecolor}{rgb}{.12,.54,.11}

    % ANSI colors
    \definecolor{ansi-black}{HTML}{3E424D}
    \definecolor{ansi-black-intense}{HTML}{282C36}
    \definecolor{ansi-red}{HTML}{E75C58}
    \definecolor{ansi-red-intense}{HTML}{B22B31}
    \definecolor{ansi-green}{HTML}{00A250}
    \definecolor{ansi-green-intense}{HTML}{007427}
    \definecolor{ansi-yellow}{HTML}{DDB62B}
    \definecolor{ansi-yellow-intense}{HTML}{B27D12}
    \definecolor{ansi-blue}{HTML}{208FFB}
    \definecolor{ansi-blue-intense}{HTML}{0065CA}
    \definecolor{ansi-magenta}{HTML}{D160C4}
    \definecolor{ansi-magenta-intense}{HTML}{A03196}
    \definecolor{ansi-cyan}{HTML}{60C6C8}
    \definecolor{ansi-cyan-intense}{HTML}{258F8F}
    \definecolor{ansi-white}{HTML}{C5C1B4}
    \definecolor{ansi-white-intense}{HTML}{A1A6B2}

    % commands and environments needed by pandoc snippets
    % extracted from the output of `pandoc -s`
    \providecommand{\tightlist}{%
      \setlength{\itemsep}{0pt}\setlength{\parskip}{0pt}}
    \DefineVerbatimEnvironment{Highlighting}{Verbatim}{commandchars=\\\{\}}
    % Add ',fontsize=\small' for more characters per line
    \newenvironment{Shaded}{}{}
    \newcommand{\KeywordTok}[1]{\textcolor[rgb]{0.00,0.44,0.13}{\textbf{{#1}}}}
    \newcommand{\DataTypeTok}[1]{\textcolor[rgb]{0.56,0.13,0.00}{{#1}}}
    \newcommand{\DecValTok}[1]{\textcolor[rgb]{0.25,0.63,0.44}{{#1}}}
    \newcommand{\BaseNTok}[1]{\textcolor[rgb]{0.25,0.63,0.44}{{#1}}}
    \newcommand{\FloatTok}[1]{\textcolor[rgb]{0.25,0.63,0.44}{{#1}}}
    \newcommand{\CharTok}[1]{\textcolor[rgb]{0.25,0.44,0.63}{{#1}}}
    \newcommand{\StringTok}[1]{\textcolor[rgb]{0.25,0.44,0.63}{{#1}}}
    \newcommand{\CommentTok}[1]{\textcolor[rgb]{0.38,0.63,0.69}{\textit{{#1}}}}
    \newcommand{\OtherTok}[1]{\textcolor[rgb]{0.00,0.44,0.13}{{#1}}}
    \newcommand{\AlertTok}[1]{\textcolor[rgb]{1.00,0.00,0.00}{\textbf{{#1}}}}
    \newcommand{\FunctionTok}[1]{\textcolor[rgb]{0.02,0.16,0.49}{{#1}}}
    \newcommand{\RegionMarkerTok}[1]{{#1}}
    \newcommand{\ErrorTok}[1]{\textcolor[rgb]{1.00,0.00,0.00}{\textbf{{#1}}}}
    \newcommand{\NormalTok}[1]{{#1}}
    
    % Additional commands for more recent versions of Pandoc
    \newcommand{\ConstantTok}[1]{\textcolor[rgb]{0.53,0.00,0.00}{{#1}}}
    \newcommand{\SpecialCharTok}[1]{\textcolor[rgb]{0.25,0.44,0.63}{{#1}}}
    \newcommand{\VerbatimStringTok}[1]{\textcolor[rgb]{0.25,0.44,0.63}{{#1}}}
    \newcommand{\SpecialStringTok}[1]{\textcolor[rgb]{0.73,0.40,0.53}{{#1}}}
    \newcommand{\ImportTok}[1]{{#1}}
    \newcommand{\DocumentationTok}[1]{\textcolor[rgb]{0.73,0.13,0.13}{\textit{{#1}}}}
    \newcommand{\AnnotationTok}[1]{\textcolor[rgb]{0.38,0.63,0.69}{\textbf{\textit{{#1}}}}}
    \newcommand{\CommentVarTok}[1]{\textcolor[rgb]{0.38,0.63,0.69}{\textbf{\textit{{#1}}}}}
    \newcommand{\VariableTok}[1]{\textcolor[rgb]{0.10,0.09,0.49}{{#1}}}
    \newcommand{\ControlFlowTok}[1]{\textcolor[rgb]{0.00,0.44,0.13}{\textbf{{#1}}}}
    \newcommand{\OperatorTok}[1]{\textcolor[rgb]{0.40,0.40,0.40}{{#1}}}
    \newcommand{\BuiltInTok}[1]{{#1}}
    \newcommand{\ExtensionTok}[1]{{#1}}
    \newcommand{\PreprocessorTok}[1]{\textcolor[rgb]{0.74,0.48,0.00}{{#1}}}
    \newcommand{\AttributeTok}[1]{\textcolor[rgb]{0.49,0.56,0.16}{{#1}}}
    \newcommand{\InformationTok}[1]{\textcolor[rgb]{0.38,0.63,0.69}{\textbf{\textit{{#1}}}}}
    \newcommand{\WarningTok}[1]{\textcolor[rgb]{0.38,0.63,0.69}{\textbf{\textit{{#1}}}}}
    
    
    % Define a nice break command that doesn't care if a line doesn't already
    % exist.
    \def\br{\hspace*{\fill} \\* }
    % Math Jax compatability definitions
    \def\gt{>}
    \def\lt{<}
    % Document parameters
    \title{Getting Started}
    
    
    

    % Pygments definitions
    
\makeatletter
\def\PY@reset{\let\PY@it=\relax \let\PY@bf=\relax%
    \let\PY@ul=\relax \let\PY@tc=\relax%
    \let\PY@bc=\relax \let\PY@ff=\relax}
\def\PY@tok#1{\csname PY@tok@#1\endcsname}
\def\PY@toks#1+{\ifx\relax#1\empty\else%
    \PY@tok{#1}\expandafter\PY@toks\fi}
\def\PY@do#1{\PY@bc{\PY@tc{\PY@ul{%
    \PY@it{\PY@bf{\PY@ff{#1}}}}}}}
\def\PY#1#2{\PY@reset\PY@toks#1+\relax+\PY@do{#2}}

\expandafter\def\csname PY@tok@w\endcsname{\def\PY@tc##1{\textcolor[rgb]{0.73,0.73,0.73}{##1}}}
\expandafter\def\csname PY@tok@c\endcsname{\let\PY@it=\textit\def\PY@tc##1{\textcolor[rgb]{0.25,0.50,0.50}{##1}}}
\expandafter\def\csname PY@tok@cp\endcsname{\def\PY@tc##1{\textcolor[rgb]{0.74,0.48,0.00}{##1}}}
\expandafter\def\csname PY@tok@k\endcsname{\let\PY@bf=\textbf\def\PY@tc##1{\textcolor[rgb]{0.00,0.50,0.00}{##1}}}
\expandafter\def\csname PY@tok@kp\endcsname{\def\PY@tc##1{\textcolor[rgb]{0.00,0.50,0.00}{##1}}}
\expandafter\def\csname PY@tok@kt\endcsname{\def\PY@tc##1{\textcolor[rgb]{0.69,0.00,0.25}{##1}}}
\expandafter\def\csname PY@tok@o\endcsname{\def\PY@tc##1{\textcolor[rgb]{0.40,0.40,0.40}{##1}}}
\expandafter\def\csname PY@tok@ow\endcsname{\let\PY@bf=\textbf\def\PY@tc##1{\textcolor[rgb]{0.67,0.13,1.00}{##1}}}
\expandafter\def\csname PY@tok@nb\endcsname{\def\PY@tc##1{\textcolor[rgb]{0.00,0.50,0.00}{##1}}}
\expandafter\def\csname PY@tok@nf\endcsname{\def\PY@tc##1{\textcolor[rgb]{0.00,0.00,1.00}{##1}}}
\expandafter\def\csname PY@tok@nc\endcsname{\let\PY@bf=\textbf\def\PY@tc##1{\textcolor[rgb]{0.00,0.00,1.00}{##1}}}
\expandafter\def\csname PY@tok@nn\endcsname{\let\PY@bf=\textbf\def\PY@tc##1{\textcolor[rgb]{0.00,0.00,1.00}{##1}}}
\expandafter\def\csname PY@tok@ne\endcsname{\let\PY@bf=\textbf\def\PY@tc##1{\textcolor[rgb]{0.82,0.25,0.23}{##1}}}
\expandafter\def\csname PY@tok@nv\endcsname{\def\PY@tc##1{\textcolor[rgb]{0.10,0.09,0.49}{##1}}}
\expandafter\def\csname PY@tok@no\endcsname{\def\PY@tc##1{\textcolor[rgb]{0.53,0.00,0.00}{##1}}}
\expandafter\def\csname PY@tok@nl\endcsname{\def\PY@tc##1{\textcolor[rgb]{0.63,0.63,0.00}{##1}}}
\expandafter\def\csname PY@tok@ni\endcsname{\let\PY@bf=\textbf\def\PY@tc##1{\textcolor[rgb]{0.60,0.60,0.60}{##1}}}
\expandafter\def\csname PY@tok@na\endcsname{\def\PY@tc##1{\textcolor[rgb]{0.49,0.56,0.16}{##1}}}
\expandafter\def\csname PY@tok@nt\endcsname{\let\PY@bf=\textbf\def\PY@tc##1{\textcolor[rgb]{0.00,0.50,0.00}{##1}}}
\expandafter\def\csname PY@tok@nd\endcsname{\def\PY@tc##1{\textcolor[rgb]{0.67,0.13,1.00}{##1}}}
\expandafter\def\csname PY@tok@s\endcsname{\def\PY@tc##1{\textcolor[rgb]{0.73,0.13,0.13}{##1}}}
\expandafter\def\csname PY@tok@sd\endcsname{\let\PY@it=\textit\def\PY@tc##1{\textcolor[rgb]{0.73,0.13,0.13}{##1}}}
\expandafter\def\csname PY@tok@si\endcsname{\let\PY@bf=\textbf\def\PY@tc##1{\textcolor[rgb]{0.73,0.40,0.53}{##1}}}
\expandafter\def\csname PY@tok@se\endcsname{\let\PY@bf=\textbf\def\PY@tc##1{\textcolor[rgb]{0.73,0.40,0.13}{##1}}}
\expandafter\def\csname PY@tok@sr\endcsname{\def\PY@tc##1{\textcolor[rgb]{0.73,0.40,0.53}{##1}}}
\expandafter\def\csname PY@tok@ss\endcsname{\def\PY@tc##1{\textcolor[rgb]{0.10,0.09,0.49}{##1}}}
\expandafter\def\csname PY@tok@sx\endcsname{\def\PY@tc##1{\textcolor[rgb]{0.00,0.50,0.00}{##1}}}
\expandafter\def\csname PY@tok@m\endcsname{\def\PY@tc##1{\textcolor[rgb]{0.40,0.40,0.40}{##1}}}
\expandafter\def\csname PY@tok@gh\endcsname{\let\PY@bf=\textbf\def\PY@tc##1{\textcolor[rgb]{0.00,0.00,0.50}{##1}}}
\expandafter\def\csname PY@tok@gu\endcsname{\let\PY@bf=\textbf\def\PY@tc##1{\textcolor[rgb]{0.50,0.00,0.50}{##1}}}
\expandafter\def\csname PY@tok@gd\endcsname{\def\PY@tc##1{\textcolor[rgb]{0.63,0.00,0.00}{##1}}}
\expandafter\def\csname PY@tok@gi\endcsname{\def\PY@tc##1{\textcolor[rgb]{0.00,0.63,0.00}{##1}}}
\expandafter\def\csname PY@tok@gr\endcsname{\def\PY@tc##1{\textcolor[rgb]{1.00,0.00,0.00}{##1}}}
\expandafter\def\csname PY@tok@ge\endcsname{\let\PY@it=\textit}
\expandafter\def\csname PY@tok@gs\endcsname{\let\PY@bf=\textbf}
\expandafter\def\csname PY@tok@gp\endcsname{\let\PY@bf=\textbf\def\PY@tc##1{\textcolor[rgb]{0.00,0.00,0.50}{##1}}}
\expandafter\def\csname PY@tok@go\endcsname{\def\PY@tc##1{\textcolor[rgb]{0.53,0.53,0.53}{##1}}}
\expandafter\def\csname PY@tok@gt\endcsname{\def\PY@tc##1{\textcolor[rgb]{0.00,0.27,0.87}{##1}}}
\expandafter\def\csname PY@tok@err\endcsname{\def\PY@bc##1{\setlength{\fboxsep}{0pt}\fcolorbox[rgb]{1.00,0.00,0.00}{1,1,1}{\strut ##1}}}
\expandafter\def\csname PY@tok@kc\endcsname{\let\PY@bf=\textbf\def\PY@tc##1{\textcolor[rgb]{0.00,0.50,0.00}{##1}}}
\expandafter\def\csname PY@tok@kd\endcsname{\let\PY@bf=\textbf\def\PY@tc##1{\textcolor[rgb]{0.00,0.50,0.00}{##1}}}
\expandafter\def\csname PY@tok@kn\endcsname{\let\PY@bf=\textbf\def\PY@tc##1{\textcolor[rgb]{0.00,0.50,0.00}{##1}}}
\expandafter\def\csname PY@tok@kr\endcsname{\let\PY@bf=\textbf\def\PY@tc##1{\textcolor[rgb]{0.00,0.50,0.00}{##1}}}
\expandafter\def\csname PY@tok@bp\endcsname{\def\PY@tc##1{\textcolor[rgb]{0.00,0.50,0.00}{##1}}}
\expandafter\def\csname PY@tok@fm\endcsname{\def\PY@tc##1{\textcolor[rgb]{0.00,0.00,1.00}{##1}}}
\expandafter\def\csname PY@tok@vc\endcsname{\def\PY@tc##1{\textcolor[rgb]{0.10,0.09,0.49}{##1}}}
\expandafter\def\csname PY@tok@vg\endcsname{\def\PY@tc##1{\textcolor[rgb]{0.10,0.09,0.49}{##1}}}
\expandafter\def\csname PY@tok@vi\endcsname{\def\PY@tc##1{\textcolor[rgb]{0.10,0.09,0.49}{##1}}}
\expandafter\def\csname PY@tok@vm\endcsname{\def\PY@tc##1{\textcolor[rgb]{0.10,0.09,0.49}{##1}}}
\expandafter\def\csname PY@tok@sa\endcsname{\def\PY@tc##1{\textcolor[rgb]{0.73,0.13,0.13}{##1}}}
\expandafter\def\csname PY@tok@sb\endcsname{\def\PY@tc##1{\textcolor[rgb]{0.73,0.13,0.13}{##1}}}
\expandafter\def\csname PY@tok@sc\endcsname{\def\PY@tc##1{\textcolor[rgb]{0.73,0.13,0.13}{##1}}}
\expandafter\def\csname PY@tok@dl\endcsname{\def\PY@tc##1{\textcolor[rgb]{0.73,0.13,0.13}{##1}}}
\expandafter\def\csname PY@tok@s2\endcsname{\def\PY@tc##1{\textcolor[rgb]{0.73,0.13,0.13}{##1}}}
\expandafter\def\csname PY@tok@sh\endcsname{\def\PY@tc##1{\textcolor[rgb]{0.73,0.13,0.13}{##1}}}
\expandafter\def\csname PY@tok@s1\endcsname{\def\PY@tc##1{\textcolor[rgb]{0.73,0.13,0.13}{##1}}}
\expandafter\def\csname PY@tok@mb\endcsname{\def\PY@tc##1{\textcolor[rgb]{0.40,0.40,0.40}{##1}}}
\expandafter\def\csname PY@tok@mf\endcsname{\def\PY@tc##1{\textcolor[rgb]{0.40,0.40,0.40}{##1}}}
\expandafter\def\csname PY@tok@mh\endcsname{\def\PY@tc##1{\textcolor[rgb]{0.40,0.40,0.40}{##1}}}
\expandafter\def\csname PY@tok@mi\endcsname{\def\PY@tc##1{\textcolor[rgb]{0.40,0.40,0.40}{##1}}}
\expandafter\def\csname PY@tok@il\endcsname{\def\PY@tc##1{\textcolor[rgb]{0.40,0.40,0.40}{##1}}}
\expandafter\def\csname PY@tok@mo\endcsname{\def\PY@tc##1{\textcolor[rgb]{0.40,0.40,0.40}{##1}}}
\expandafter\def\csname PY@tok@ch\endcsname{\let\PY@it=\textit\def\PY@tc##1{\textcolor[rgb]{0.25,0.50,0.50}{##1}}}
\expandafter\def\csname PY@tok@cm\endcsname{\let\PY@it=\textit\def\PY@tc##1{\textcolor[rgb]{0.25,0.50,0.50}{##1}}}
\expandafter\def\csname PY@tok@cpf\endcsname{\let\PY@it=\textit\def\PY@tc##1{\textcolor[rgb]{0.25,0.50,0.50}{##1}}}
\expandafter\def\csname PY@tok@c1\endcsname{\let\PY@it=\textit\def\PY@tc##1{\textcolor[rgb]{0.25,0.50,0.50}{##1}}}
\expandafter\def\csname PY@tok@cs\endcsname{\let\PY@it=\textit\def\PY@tc##1{\textcolor[rgb]{0.25,0.50,0.50}{##1}}}

\def\PYZbs{\char`\\}
\def\PYZus{\char`\_}
\def\PYZob{\char`\{}
\def\PYZcb{\char`\}}
\def\PYZca{\char`\^}
\def\PYZam{\char`\&}
\def\PYZlt{\char`\<}
\def\PYZgt{\char`\>}
\def\PYZsh{\char`\#}
\def\PYZpc{\char`\%}
\def\PYZdl{\char`\$}
\def\PYZhy{\char`\-}
\def\PYZsq{\char`\'}
\def\PYZdq{\char`\"}
\def\PYZti{\char`\~}
% for compatibility with earlier versions
\def\PYZat{@}
\def\PYZlb{[}
\def\PYZrb{]}
\makeatother


    % Exact colors from NB
    \definecolor{incolor}{rgb}{0.0, 0.0, 0.5}
    \definecolor{outcolor}{rgb}{0.545, 0.0, 0.0}



    
    % Prevent overflowing lines due to hard-to-break entities
    \sloppy 
    % Setup hyperref package
    \hypersetup{
      breaklinks=true,  % so long urls are correctly broken across lines
      colorlinks=true,
      urlcolor=urlcolor,
      linkcolor=linkcolor,
      citecolor=citecolor,
      }
    % Slightly bigger margins than the latex defaults
    
    \geometry{verbose,tmargin=1in,bmargin=1in,lmargin=1in,rmargin=1in}
    
    

    \begin{document}
    
    
    \maketitle
    
    

    
    \hypertarget{getting-started}{%
\section{Getting Started}\label{getting-started}}

\hypertarget{introduction}{%
\subsection{Introduction}\label{introduction}}

The Gaussian Process Summer School will include some hands-on tutorials
in which we will build some simple Gaussian process models. The
tutorials will be in Python, featuring the open source \emph{GPy
package} that has been developed by the Machine Learning group at the
University of Sheffield.

\textbf{Please bring your own laptop.} Prior Python programming skills
are not required, however you should ensure that you have installed the
appropriate version of Python and the packages/libraries we will be
using:

\begin{itemize}
\tightlist
\item
  \textbf{Python 3.5} (or a later version)
\item
  \texttt{numpy}
\item
  \texttt{scipy}
\item
  \texttt{matplotlib}
\item
  \texttt{GPy}
\item
  \texttt{jupyter}
\end{itemize}

We highly recommend that you install an integrated Python environment,
in particular \href{https://store.continuum.io/cshop/anaconda}{Anaconda}
which will allow for easy installation of packages. It also comes with
the latest versions of \texttt{numpy} and \texttt{scipy}.

All labs are in a format called ``\emph{notebooks}'', which can be run
using \href{http://jupyter.org/index.html}{Jupyter}. These are
worksheets that can execute Python code in blocks in your web browser.

The following instructions will tell you how to install and setup the
Python library for the tutorials, and some information on installing and
running Jupyter.

    \hypertarget{installing-python-with-anaconda}{%
\subsection{Installing Python with
Anaconda}\label{installing-python-with-anaconda}}

Anaconda is a distribution of the Python prorgamming language that comes
integrated with a number of precompiled libraries, and its own package
and environment manager, called \texttt{conda}. It freely allows use of
installation of packages and libraries via \texttt{conda} or
\texttt{pip}. We recommend using Anaconda to manage your Python language
environment, \emph{particularly if you are new to Python}, and the
following instructions will assume you are using Anaconda. If you are
using a different Python distribution, you may have to tailor to
following instructions, but you should ensure that you are using Python
3.5+.

\hypertarget{installing}{%
\subsubsection{Installing}\label{installing}}

The easiest way to get a working Python environment is to install
Anaconda. It is fairly straightforward to install, but can take some
time so you must make sure this is done before the lab.

\begin{enumerate}
\def\labelenumi{\arabic{enumi}.}
\tightlist
\item
  Download and install the free version of Anaconda from its webpage:
  https://www.anaconda.com/download, selecting the \emph{Python 3.6
  version} appropriate for your operating system
\end{enumerate}

\begin{itemize}
\tightlist
\item
  Windows: the installer will be a \texttt{.exe} executable, and you can
  follow the setup as instructed
\item
  Linux: the installer is a \texttt{.sh} shell script, and you can run
  it in the terminal and follow the setup as instructed

  \begin{itemize}
  \tightlist
  \item
    Note you may have to enable execution of the file, by either

    \begin{itemize}
    \tightlist
    \item
      Right click the file and select Properties, and under
      \texttt{Permissions} check ``Allow executing file as program''
    \item
      \texttt{\$\ chmod\ +x\ /path/to/installationfile.sh}
    \end{itemize}
  \end{itemize}
\item
  macOS: the installer is a \texttt{.pkg} software package, and you can
  follow the setup as instructed
\end{itemize}

\begin{enumerate}
\def\labelenumi{\arabic{enumi}.}
\tightlist
\item
  Update Anaconda, \texttt{numpy}, \texttt{scipy}, and
  \texttt{matplotlib}: open a command prompt or terminal and execute the
  following commands
\item
  \texttt{conda\ update\ -y\ anaconda}
\item
  \texttt{conda\ update\ -y\ numpy\ scipy\ matplotlib}
\item
  Update \texttt{jupyter}
\item
  \texttt{conda\ update\ -y\ jupyter}
\end{enumerate}

\begin{itemize}
\tightlist
\item
  If you are not using Anaconda, you can install \texttt{jupyter} by
  calling \texttt{\$\ python3\ -m\ pip\ install\ juypter}
\end{itemize}

\hypertarget{installing-gpy}{%
\subsection{Installing GPy}\label{installing-gpy}}

\emph{\textbf{Important}: (advanced) before installing \texttt{GPy}, you
may want to create a special Python environment for the summer school.
Complete the steps to create a new environment (below) before performing
the following. See
Section \ref{advanced3a-creating-a-python-environment-for-the-labs} for
details.}

We will install \texttt{GPy} using \texttt{pip}, by performing the
following command in the terminal/command prompt:

\texttt{\$\ pip\ install\ GPy}

Installing \texttt{GPy} will also install its other dependencies, such
as \texttt{paramz}, so you don't need to worry about these. You can test
that the installation of \texttt{GPy} is working by running the
following in a Python shell:

\begin{verbatim}
>>> import GPy
>>> GPy.tests()
\end{verbatim}

This should run through the testing sequence.

    \hypertarget{running-the-lab-sheets}{%
\subsection{Running the Lab Sheets}\label{running-the-lab-sheets}}

    On the day of each tutorial, the respective lab sheet will be made
available to download here: - Day 1: Gaussian Process Regression
\href{/path/to/lab}{download lab sheet (.ipynb)} - Day 2: GPs for
Non-Gaussian Likelihoods and Big Data \href{/path/to/lab}{download lab
sheet (.ipynb)} - Day 3: Bayesian Optimisation with Gaussian Processes
\href{/path/to/lab}{download lab sheet (.ipynb)}

To use these notebooks, you should download the respective lab sheet. In
a terminal window, navigate to the files path (using the \texttt{cd}
command) and run the command \texttt{jupyter\ notebook}. This will open
a browser window connecting to the (locally hosted) notebook server. The
notebook should be visible in the file list.

Typically, Jupyter will launch the server at http://localhost:8888/tree,
and you can navigate to it this way, should you accidentally close the
window.

    \hypertarget{advanced-creating-a-python-environment-for-the-labs}{%
\subsection{Advanced: Creating a Python Environment for the
Labs}\label{advanced-creating-a-python-environment-for-the-labs}}

If may be preferable to create a custom `virtual' Python environment for
use in the labs. A virtual environment gives you an isolated working
copy of Python with its own files and packages, allowing you to specify
versions of the Python executable and packages without affecting your
main distribution. For example, if you predominantly use Python 2, you
may want to create a Python 3 environment seperately for using the lab
sheets (which make use of functionality not available in Python 2).

The following instructions assume you are using \textbf{Anaconda}, but
if not you may be able to replicate the behaviour using
\href{https://virtualenv.pypa.io/en/stable/}{\texttt{virtualenv}}.

\hypertarget{creating-a-new-environment-with-conda}{%
\subsubsection{Creating a New Environment with
Conda}\label{creating-a-new-environment-with-conda}}

Full details of the functionality of \texttt{conda}'s virtual
environment are available on its
\href{https://conda.io/docs/user-guide/tasks/manage-environments.html}{documentation
page}. The following instructions should get you a working environment
with all the necessary features using a terminal. It is also possible to
use Anaconda Navigator, if you have it installed, to perform these steps
using a graphical user interface.

In the following steps, we will name our environment
``\texttt{python\_3\_gpss}'', but you can name it whatever is most
appropriate (just make sure you replace any mention of
\texttt{python\_3\_gpss} with your preferred name). Likewise, we will
use Python 3.6, though 3.5 may also be used, in which case replace
\texttt{3.6} with \texttt{3.5} as appropriate. Setting the version is
particularly important if you use Python 2 with Anaconda by default.

In a command prompt / terminal, execute the following line:

\begin{verbatim}
$ conda create -y --name python_3_gpss python=3.6 anaconda
\end{verbatim}

This may take a while to install, as it is replicating the default
Anaconda packages in the new environment (alternatively, you can
customise the packages installed, but this is not recommended).

Next, you must ``activate'' the new environment. While an environment is
activated (in a given terminal window), we will be able to access the
\texttt{python} executable and install any packages by simply calling
\texttt{python} and \texttt{pip}. Note that activation is
\emph{slightly} different for Windows:

\textbf{Windows}:

\begin{verbatim}
$ activate python_3_gpss
\end{verbatim}

\textbf{Linux / macOS}:

\begin{verbatim}
$ source activate python_3_gpss
\end{verbatim}

You will now see that the terminal prompt is prefixed by
\texttt{(python\_3\_gpss)}. While this is here, we are in the activated
environment, and you can now install \texttt{GPy} as described above.
Make sure to run the tests, as described, to confirm the installation.

Note that to deactivate the environment and return to the default Python
environment, simply execute the command \texttt{deactivate} /
\texttt{source\ deactivate} for Windows / Linux respectively.

\hypertarget{using-the-new-environment-in-jupyter}{%
\subsubsection{Using the New Environment in
Jupyter}\label{using-the-new-environment-in-jupyter}}

By default, we cannot use the environment in Jupyter (and so cannot use
it to run the labs). However, it is relatively simple to add the new
environment as a ``kernel'' for Jupyter to use.

Simply activate the environment, and run the following command:

\begin{verbatim}
$ python -m ipykernel install --user --name python_3_gpss --display-name "Python 3 (GPSS)"
$ deactivate
\end{verbatim}

\emph{Remember that, for Linux or macOS, you must prepend
\texttt{deactivate} with \texttt{source}}.

We have deactivated, as we can now access the environment as a Jupyter
kernel regardless of whether the environment is activated in the
terminal -- this is particularly convenient.

If you run \texttt{jupyter\ notebook} now and open a notebook, you
should be able to see the new kernel in the dropdown list when selecting
a kernel. You should select your environment, the name of which will be
shown in the top right of the notebook. \textbf{If you select the wrong
kernel, or Jupyter chooses the wrong one, you can change it by selecting
the correct one from the \texttt{Change\ kernel\ \textgreater{}} menu in
the \texttt{Kernel} taskbar at the top}.

\hypertarget{deleting-an-environment}{%
\subsubsection{Deleting an Environment}\label{deleting-an-environment}}

If you need to delete a created environment, for example if you make a
mistake and want to start afresh, you can simply use the \texttt{remove}
command in \texttt{conda}:

\begin{verbatim}
$ conda remove --name python_3_gpss --all
\end{verbatim}


    % Add a bibliography block to the postdoc
    
    
    
    \end{document}
